\documentclass[12pt]{article}

% Page layout
\usepackage{geometry}
\geometry{
  top=1in,
  bottom=1in,
  left=1in,
  right=1in,
  headheight=3ex,
  headsep=4ex,
}

% Header/footer
\usepackage{fancyhdr}
\pagestyle{fancy}
\fancyhf{}
\renewcommand{\footrulewidth}{0.4pt}
\lhead{CSC311}
\rhead{Machine Learning Project}
\rfoot{Page \thepage}
\cfoot{v1.0}
\lfoot{\copyright Alice Gao 2025}

% Hyperlinks
\usepackage{hyperref}
\hypersetup{
    colorlinks=true,
    linkcolor=blue,
    filecolor=magenta,
    urlcolor=blue,
}

% Colors (for environments)
\usepackage{xcolor}

% Formatting
\setlength{\parskip}{\baselineskip}%
\setlength{\parindent}{0pt}%

% Custom environments
\newenvironment{answer}[1][]{
  \color{blue}\textbf{Answer:}
}{}

\newenvironment{alice}[1][]{
  \color{black}\textbf{Tip:}
}{}


\title{CSC311 Machine Learning Project Group Contract}
\author{}
\date{}


\begin{document}

\maketitle

This group contract helps you set and communicate your expectations for working together on the project. If any issues come up, the course staff will use this contract to help resolve them.

Discuss the following questions as a group and fill out the answers below.

\begin{enumerate}
\item {\bf Group Members:} Write down the full names of your group members and UTORids.

\begin{answer}

Yousef Noureddin Ibrahim (ibrah857)

Yige Wu (wuyige3)

Emma Chow (chowemm1)

Yanzun Jiang (jian1157)
\end{answer}

\item {\bf Goals and Expectations} What does each member want to achieve from this project? Are your goals aligned as a group?

\begin{answer}

All group members hope to achieve an 85+ in the course, learning more about the fundamentals of machine learning, and how to employ them effectively for future courses, and career opportunities.
\end{answer}

\item {\bf Group Roles:} What roles are necessary for the success of your project? Who will be assigned to each role? Consider each member's strengths and weaknesses.

\begin{answer}

Potential Roles: Project Manager (deadlines, meetings), Code Developer (GitHub, model design), Model Analyzer (evaluation, metrics), Documentation Lead (report)

Assignments: Emma is the Project Manager. Yige is the Code Developer. Yanzun is the Model Analyzer. Yousef is the Documentation Lead

Collaboration: All members code and contribute to all stages. Roles define primary responsibility, not exclusive work. We will share the work equally
\end{answer}

\item {\bf Communication:} How will you communicate? What are your expectations for response times? How often will you meet, and for how long? What technology will you use?

\begin{alice}

Schedule a mandatory weekly meeting to stay on track and address issues promptly.
\end{alice}

\begin{answer}

- Through Discord

- On Saturday 3-4 pm (additional meetings if needed)

- Respond within 8 hours

- GitHub, Google Docs, ...
\end{answer}

\item {\bf Preparation for Meetings:} What should members do before each weekly meeting?

\begin{alice}

Create a list of to-do items at the end of each meeting and review them at the next meeting.

\end{alice}

\begin{answer}

- Complete the tasks (to-do lists) in the last meeting

- Summarize questions during working on the project

- Evaluating current progress (difficulties encountered during work) and how things are going

- Bringing up any concerns if present

- Planning on future tasks
\end{answer}


\item {\bf Meeting Conduct:} What are your expectations for attitudes and responses during meetings? How will you manage turn-taking and ensure everyone contributes? How will you make decisions?

\begin{answer}

Expectation \& Attitude:

- Create a growth environment

- Be respectful and non-judgmental

Contribution:

- Assign tasks equally for everyone

- Everyone should speak in the meeting

Making decisions:

- Majority vote held in the group chat

- Discuss consensus during meetings
\end{answer}

\begin{alice}

Structure meetings effectively and decide on a method for resolving disagreements, such as consensus or majority vote.
\end{alice}

\item {\bf Work Structure:} How will you structure the work? Will most of it be done during or outside of meetings? How will you assign responsibilities?

\begin{answer}

- Most of the work will be done outside of the meetings, only check-ins during meetings

- Will ensure everyone has an equal workload

- Will structure/assign the work according to each person's strength \& weakness / preferences
\end{answer}

\begin{alice}

Use a divide-and-conquer approach, ensuring an equal workload for all members.
\end{alice}

\item {\bf Submission of Deliverables:} How will you submit deliverables? Will all members review the submission before it's finalized? When should the write-up be ready for review?

\begin{answer}

- Collaborate through GitHub

- Everyone is able to and should review the submission

- The write-up should be ready one week before the due date

- After all members think the work is good to go, one person submit it
\end{answer}

\begin{alice}

Communicate struggles early and work together to find solutions.
\end{alice}

\item {\bf Handling Surprises:} How will you deal with unexpected challenges? What should a member do if they can't deliver on a promise? How will the group respond?

\begin{answer}

- Be fully transparent with any struggles

- Other one try to fill in, and that person take on extra work the following week

- All group members should be understanding
\end{answer}

\begin{alice}

Communicate struggles early and work together to find solutions.
\end{alice}

\item {\bf Conflict Resolution:} How will you handle conflicts? How can a member signal an issue? How will the group respond?

\begin{answer}

- Address issues directly and professionally

- Set up a call to address the issue

- Focus on finding a solution, not assigning blame

- Immediately contact the course staff if we can't resolve it internally
\end{answer}

\begin{alice}

Establish a protocol for resolving conflicts and seek help from course staff if needed.
\end{alice}

\end{enumerate}

\end{document}
